%---------------------------------------------------------------------------%
%								                    ABSTRACT                       					%
%---------------------------------------------------------------------------%
\begin{abstractpage}

The global rate of waste production has consistently outpaced the world's ability to manage and remediate it. Specifically, global consumption of raw materials, unrenewable energy sources, and disposal of electronic goods have contaminated water sources with heavy metals causing enviornmental damage and public health concerns.
% causing tremendous environmental damage while also endangering public health with heavy metal poisoning.
Despite the urgent need to contain and remove metals from the environment, there still does not exist robust and complete remediation technologies. Physicochemical technologies like chemical precipitation, absorption, and ion-exchange lack the specificity for metal capture, produce their own secondary-waste in the form of chemical by-products or sludge, and have a high cost barrier requiring development of dedicated infrastructure and technical expertise.
% Despite the urgent need to contain and remove metals from the environment, there still does not exist robust and complete remediation technologies. Currently the technologies employed are synthetic, relying on physical and chemical separation of toxins.
% Unfortunately, three of the most commonly used physicochemical strategies: chemical precipitation, absorption, and ion-exchange, lack the specificity for metal capture, produce their own secondary-waste in the form of chemical by-products or sludge, and have a high cost barrier requiring development of dedicated infrastructures and technical expertise to manage and operate.

Instead, this work investigates biologically-derived strategies for managing waste, technologies also known as bioremediation. Principles from chemical precipitation, absorption, and ion-exchange were analogously designed in \textit{S. cerevisae}---the common baker's yeast. The three analogies were: engineering yeast sulfur metabolic pathways for controlled metal sulfide precipitation; designing new metal trafficking schemes using membrane metal transporters; and engineering supramolecular forming proteins for yeast-protein metal chelation and sequestration. For all methods, metal removal were between 50--90\% efficiency for heavy metals such as Cu, Cd, Hg, and Pb. Furthermore, 2--4 rounds of processing eliminated almost 100 $\mu$M of metal, 100--1000 fold greater than EPA toxicity thresholds. Strategies to retrieve and recycle captured metals were also investigated, such as precipitating metal sulfide crystals onto the yeast surface, compartmentalizing metals into the yeast vacuole, or sedimenting bound metals into cell-protein complexes.

Relying on yeast takes advantage of their autonomous growth, ease of engineering, and its ubiquitous presence in the household and consumer market.
The purpose of this work was to show that the same yeast used for brewing and baking can be harnessed for clean water applications.
% The technologies behind the growth, storage, and distribution of yeast have allowed it to touch all areas of the globe. Therefore yeast provided a technological marriage between the ability to engineer novel metal removing capabilities, and the ability to scale, distribute, and store yeast through the mature yeast market.

\end{abstractpage}
